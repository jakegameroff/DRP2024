\documentclass[reqnom, 12pt]{amsart}
\usepackage{amsthm, amssymb, amsmath, mathtools, braket}
\usepackage{setspace}
\usepackage[shortlabels]{enumitem}
\usepackage{lipsum} 
%\usepackage{MnSymbol,wasysym}
\usepackage{comment}
%\setlist{nolistsep}
%\onehalfspacing

%%%—definitions, theorems, propositions, etc.—%%%
\usepackage[dvipsnames]{xcolor}
\usepackage{thmtools}
\usepackage[framemethod=TikZ]{mdframed}
\mdfsetup{skipabove=1em,skipbelow=0em, rightmargin=1em, innertopmargin=5pt, innerbottommargin=10pt}

\declaretheoremstyle[headfont=\bfseries, bodyfont=\itshape]{definition}
\declaretheoremstyle[headfont=\bfseries\color{black}, bodyfont=\itshape]{thm}
\declaretheoremstyle[headfont=\bfseries\color{black}, bodyfont=\itshape]{prop}
\declaretheoremstyle[headfont=\bfseries, bodyfont=\itshape]{cor}
\declaretheoremstyle[headfont=\itshape, bodyfont=\normalfont]{obs}
\declaretheoremstyle[headfont=\bfseries, bodyfont=\normalfont]{example}

\declaretheorem[numberwithin=section, style=definition, name=Definition]{definition}
\declaretheorem[numberwithin=section, style=thm, name=Theorem]{theorem}
\declaretheorem[numberwithin=section, style=prop, name=Proposition]{proposition}
\declaretheorem[style=obs, name=Observation]{observation}
\declaretheorem[numberwithin=section, style=cor, name=Lemma]{lemma}
\declaretheorem[numberwithin=section, style=cor, name=Corollary]{corollary}
\declaretheorem[style=example, name=Example]{example}
\declaretheorem[numberwithin=section, style=example, name=Remark]{remark}

\makeatletter
\def\@settitle{\begin{center}%
  \baselineskip14\p@\relax
    %\bfseries
    \normalfont\Large%<- NEW
\@title
  \end{center}%
}
\makeatother

\title{Ramsey Properties of Isometric $k$-Colorings}
\author{Jake R. Gameroff}
\begin{document}
\begin{abstract}
	A map \( \chi : [X]^{k} \to [k] \) on a metric space \( X \) is an isometric \( k \)-coloring if \( \chi(A_1) = \chi(A_2) \) for every pair \( A_1,A_2 \) of isodiametric \( k \)-element subsets of \( X \). A free ultrafilter \( \mathcal{F}  \) is called metrically Ramsey if for every isometric coloring \( \chi \) there is a set \( A \in \mathcal{F}  \) such that \( [A]^{k}  \) is \( \chi \)-monochrome. Protasov and Protasova [1] prove that every infinite ultrametric space \( X \) contains a sequence \( (x_{n}) \) such that every free ultrafilter containing \( (x_{n}) \) is metrically Ramsey when \( k = 2 \). We strengthen this result to hold for every \( k \in \mathbb{N}  \).
\end{abstract}
\maketitle
%! TeX root: ../main.tex
\section{Introduction}
Ramsey Theory explores the underlying structure emerging in ``large enough" complex systems. For example, Frank Ramsey proved in \cite{ramsey:1930} that for each \( k \in \mathbb{N}  \) there is a sufficiently large \( n \in \mathbb{N}  \) such that in any red-blue coloring of the edges of the complete graph \( K_{n} \) there is a set of \( k \) vertices joined by edges of the same color.

Another seminal result is Van der Waerden's theorem, which states that for all positive integers \( r, k \in \mathbb{N}  \) there is a large enough \( n \in \mathbb{N}  \) such that if we color the integers in \( [n] \coloneqq \{ 1,2,\hdots ,n \}  \) using \( k \) colors, one can always find a set of \( r \) monochromatic integers in arithmetic progression \cite{waerden:1927}.

Motivated by these classical results, we study the structural properties of infinite spaces using a Ramsey-theoretic lens. In particular, we will color a class of subsets of the space and search for a set whose subsets in this class are all the same color. We formalize this as follows.

Fix an infinite metric space \( (X,d) \) and let \( k \in \mathbb{N}  \) be a postive integer. For a family \( \mathcal{A} \) of subsets of \( X \), a \emph{coloring} of \( \mathcal{A}  \) is any mapping \( \chi : \mathcal{A} \to [k] \). We would like to find a set \( M \subseteq X \) and a color \( \varphi  \in [k] \) such that \( \chi(N) = \varphi  \) for every subset \( N \subseteq M \) with \(N \in \mathcal{A}  \). We then say that \( M \) is \emph{monochrome} with respect to \( \mathcal{A}  \). A natural first choice is \( \mathcal{A} = [X]^{2}  \), the class of two-element subsets of \( X \).

In this context, the ``large" objects in \( X \) which we will work with are free ultrafilters. A \emph{filter} \( \mathcal{F}  \) on \( X \) is a collection of subsets of \( X \) satisfying the following for all subsets \( A, B \subseteq X \):
\begin{enumerate}[leftmargin=1.2cm]
	
	\item If \( A \in \mathcal{F} \) and \( A \subseteq B \) then \( B \in \mathcal{F}  \);
	\item If \( A, B \in \mathcal{F}  \) then \( A \cap B \in \mathcal{F}  \);
	\item \( \emptyset  \notin \mathcal{F}  \) and \( X \in \mathcal{F}  \).
\end{enumerate}
A filter \( \mathcal{F}  \) is called an \emph{ultrafilter} if it is not properly contained in another filter. We call \( \mathcal{F}  \) \emph{free} if \( \medcap \mathcal{F} = \emptyset  \). Intuitively, ultrafilters are maximal filters and free filters are ``spread out" throughout the space, so we view \emph{free ultrafilters} as ``large" objects in \( X \).

Given that these are the large objects in focus, it is natural to ask whether there is a free ultrafilter \( \mathcal{F}  \) such that for every coloring of \( [X]^{2}  \) there is a monochrome set \( M \in \mathcal{F}  \). It turns out that this question is undecidable in ZFC even with \( X = \mathbb{N}  \), though the statement is true if we accept the continuum hypothesis \cite{protasov:2017}. Consequently, we must define more restrictive classes of colorings.

Towards this end, Protasov and Protasova introduce the following theory in \cite{protasov:2017}. A coloring \( \chi : [X]^{2} \to [k]  \) of the two-element subsets of \( X \) is called \emph{isometric} if \( \chi (\{ x_1, y_1 \} ) = \chi (\{ x_2, y_2 \} ) \) whenever \( d(x_1, y_1) = d(x_2, y_2) \). A free ultrafilter \( \mathcal{F}  \) is called \emph{metrically Ramsey} if for every isometric coloring of \( [X]^{2}  \) there is a monochrome set \( M \in \mathcal{F}  \).

It turns out that in the particular case of ultrametric spaces, metrically Ramsey free ultrafilters are not too hard to construct. Recall that an \emph{ultrametric} \( d \) on a set \( X \) is a metric satisfying the \emph{ultrametric inequality}, which states that for all \( x,y,z \in X \) \[ d(x,y) \leq \max \{ d(x,z), d(z,y) \}.  \] Protasov and Protasova leverage the properties of the ultrametric to prove the following theorem \cite{protasov:2017}.

\begin{theorem}
\label{thm:1}
Every infinite ultrametric space \( X \) contains a sequence \( (x_{n}) \) such that any free ultrafilter \( \mathcal{F}  \) with \( (x_{n}) \in \mathcal{F} \) is metrically Ramsey.
\end{theorem}

As a follow up, one may ask if similar structure exists when coloring a larger class of subsets of \( X \). The positive answer to this question is the keynote of this paper. In this connection, we will analyze the family \( \Gamma_{X}  \) of all compact subsets of \( X \).

We generalize isometric colorings along these lines. The map \( \chi : \Gamma_{X}  \to [k] \) is called a \emph{diametric coloring} if \( \chi (A_1) = \chi (A_2) \) for every pair \( A_1, A_2 \) of compact subsets of \( X \) with \( \operatorname{diam} A_1 = \operatorname{diam} A_2  \). Accordingly, a set \( M \subseteq X \) is monochrome if its compact subsets are the same color. 

Given this, we say that a free ultrafilter \( \mathcal{F}  \) on \( X \) is \emph{diametrically Ramsey} if for every diametric coloring \( \chi \) there is a monochrome set \( M  \in \mathcal{F}  \). Since finite sets are compact, \( \Gamma_{X}  \) contains \( [X]^{2}  \) so that every diametric coloring is isometric.

In this context, our main result is the following.
\begin{theorem}
\label{main:result}
Every infinite ultrametric space \( X \) contains a sequence \((x_{n})\) such that every free ultrafilter \( \mathcal{F}  \) with \( (x_{n}) \in \mathcal{F}  \) is diametrically Ramsey.
\end{theorem}

Building towards the main result, the following two sections review some elementary properties of ultrametric spaces and filters. We then prove Theorem \ref{main:result} in Section \ref{sec:4}.

%! TeX root: ../main.tex
\section{Ultrametric Analysis}
In this short section, we provide some interesting examples of ultrametric spaces and survey some of their fundamental properties.
\subsection{Examples: The space \( \mathbb{N} ^{\mathbb{N} } \), graphs, and \( \varepsilon  \)-chains} The simplest example of an ultrametric on a set $X$ is the discrete metric $d$, where $d(x,y)$ is 1 if $x \neq y$ and 0 otherwise. Another simple example is $(\mathbb{N}, d)$, where \( d(n,m) = \max \{ 1 + 1/n , 1 + 1/m \}  \) if \( n \neq m \) and \( d(n, m) = 0 \) otherwise. This metric was first constructed by (guy) in (year) to (task) [3]. 

However, there are much more interesting constructions. In this regard, we will construct ultrametrics on the Baire space $\mathbb{N} ^{\mathbb{N} }$, connected graphs, and general uniformly disconnected metric spaces.

\subsubsection{The Baire space} We will first discuss the \emph{Baire space} \( \mathbb{N} ^{\mathbb{N} }  \), which is the space of all sequences of natural numbers.

For two distinct sequences \( x =  (x_{n}) \), \( y = (y_{n} ) \) in \( \mathbb{N}  \), we define \( m(x,y) = \min \{ k \in \mathbb{N} : x_{k} \neq y_{k}  \}  \) to be the first index at which \( x \) and \( y \) do not coincide. For distinct sequences \( x,y \in \mathbb{N} ^{\mathbb{N} }  \), set \( d(x,y) = m(x,y)^{-1}  \) with \( d(x,y) = 0 \) if \( x=y \). Then \( d \) is an ultrametric on \( \mathbb{N} ^{\mathbb{N} }  \).
\begin{proof}
The symmetry and non-negativity of \( d \) is immediate. Furthermore, \( d(x,x) = 0 \) by definition, and if \( d(x,y) = 0  \) then we must have \( x = y \) since \( m(x,y) \geq 1 \). To prove the strong triangle inequality for \( d \), fix \( x,y,z \in \mathbb{N} ^{\mathbb{N} }  \). We assume \( x,y,z \) are all distinct sequences as the result is immediate otherwise. Observe that
\begin{align*}
	&d(x,y) \leq \max \{ d(x,z), d(z,y) \} \\
	&\Leftrightarrow m(x,y)^{-1} \leq \max \{ m(x,z)^{-1} , m(z,y)^{-1}  \}  \\
	&\Leftrightarrow m(x,y)^{-1}  \leq \min \{ m(x,z), m(z,y) \} ^{-1} \\
	&\Leftrightarrow m(x,y) \geq \min \{ m(x,z), m(z,y) \}.
\end{align*}
Clearly \( m(x,y) \geq \min \{ m(x,z), m(z,y) \}  \): if \( \ell = m(x,y) \) then for each \( k \in \{ 1, 2, \hdots , \ell - 1 \} \) the terms \( x_{k} = y_{k}  \), so if \( m(x,z) \geq \ell + 1 \) and \( m(z,y) \geq \ell + 1 \) then \( x_{\ell} = z_{\ell} = y_{\ell}   \) so that \( m(x,y) \geq \ell + 1 \) is a contradiction. Therefore, \( d \) is an ultrametric.
\end{proof}
\subsubsection{Graphs} (the following is due to (author), with intuition, ...) Let \( G \) be a connected graph with positive edge-weights. For an edge \( e \) in \( G \), let \( w(e) \) denote its weight. Given a walk \( x = (v_1, v_2, \hdots , v_{n}) \) in \( G \), we will denote by \( e_{x} \) an edge in the walk with maximum weight.

We say that a walk \( x \) between two vertices \( u , v \) is a \emph{minimax walk} if there is no other walk \( x' \) between \( u,v \) whose max-weight edge is lighter than \( e_{x}  \). Equivalently, \( w(e_{x}) \leq w(e_{x'}) \) for every walk \( x' \) between \( u,v \).

We can use this notion to define an ultrametric on \( V(G) \). Specifically, given two distinct vertices \( u,v \in V(G) \) with minimax walk \( x \), we set \( \ell(u, v) = w(e_{x}) \), with \( \ell(u,u) = 0 \). The proof that \( \ell \) is an ultrametric is as follows.

\begin{proof}
That \( \ell \) is symmetric and non-negative is immediate. By definition, \( \ell(u, u) = 0 \) and if \( \ell(u, v) = 0 \) then \( u = v \), as otherwise there is an edge in \( G \) with weight 0 even though its edges have only positive weights.

To prove the ultrametric inequality, fix \( u, v, w \in V(G) \). We may assume that \( u,v,w \) are distinct as otherwise the claim is immediate. Let \( x_{u,w} = (v_1, v_2, \hdots , v_{m}) \) and \( x_{w,v} = (v_{m}, v_{m+1} , \hdots , v_{n}) \) denote minimax walks between \( u,w \) and \( w,v \) respectively, and let \( x = (v_1, v_2, \hdots , v_{n}) \) be their union. Then the max-weight edge in \( x \) has weight \( \max \{ \ell (u, w), \ell (w,v) \}  \). Since \( x \) is a walk between \( u,v \), we have \( \ell (u, v) \leq \max \{ \ell (u,w), \ell (w,v) \}  \), completing the proof.
\end{proof}

\subsubsection{Uniformly disconnected spaces} (cite def) Let \( (X,d) \) be any metric space and \( \varepsilon > 0 \) be given. An \emph{\( \varepsilon  \)-chain} between the pair \( x,y \) of distinct points in \( X \) is a finite sequence \( x = x_0 , x_1, \hdots , x_{n} = y  \) such that \[\max_{1 \leq i \leq n}  d(x_{i-1} , x_{i})  \leq \varepsilon \cdot d(x,y). \] In this case, we say that \( x, y \) are \( \varepsilon  \)-connected. Observe that if \( x, y \) are \( \varepsilon  \)-connected then they are \( \varepsilon ' \)-connected for every \( \varepsilon ' \geq \varepsilon  \); and if \( x, y \) can not be \( \varepsilon  \)-connected then they can not be \( \varepsilon ' \)-connected for every \( \varepsilon ' \leq \varepsilon  \).


The space \( X \) is called \emph{uniformly disconnected} if there is an \( \varepsilon > 0 \) such that no two points in \( X \) can be \( \varepsilon  \)-connected. It is not hard to prove that uniform disconnectivity is stronger than total disconnectivity. To give some intuition, we note that the Cantor set \( \mathcal{C}  \) is uniformly disconnected and the set \( \{ 1/n : n \in \mathbb{N}  \}  \) is not [tb].


Let \( (X,d) \) be a uniformly disconnected metric space. For \( x,y \in X \), let \( c(x,y) \) be the infimum over all \( \varepsilon > 0 \) such that \( x \) and \( y \) are \( d(x,y)^{-1} \cdot \varepsilon  \)-connected. Then \( c \) is an ultrametric on \( X \).
\begin{proof}
Clearly \( c \) is symmetric and non-negative. If \( x = y \) then clearly \( x,y \) are \( \varepsilon  \)-connected for every \( \varepsilon > 0 \) so that \( c(x,y) = 0 \). On the other hand, if \( c(x,y) = 0  \) then \( x = y \), otherwise for each \( \varepsilon > 0 \) the points \( x,y \) are \( \varepsilon  \)-connected, so \( X \) is not uniformly disconnected, a contradiction.

The last thing to prove is the strong triangle inequality for \( c \). To this end, fix \( x,y,z \in X \) and let \( \varepsilon > 0 \) be given. We may assume the points \( x,y,z \) are distinct, otherwise the claim is immediate.

By definition of the infimum, there exist \( \gamma_1, \gamma_2 > 0 \) with \( \gamma_1 \leq c(x,z) + \varepsilon  \) and \( \gamma_2 \leq c(z,y) + \varepsilon  \) such that \( x,z \) are \( d(x,z)^{-1} \cdot \gamma_1 \)-connected and \( z,y \) are \(d(z,y)^{-1} \cdot \gamma_2 \)-connected. Hence we obtain sequences
\begin{itemize}[leftmargin=1.2cm]
	\item \( x = x_0, x_1, \hdots x_{m} = z  \) with \( \max_{1 \leq i \leq m} d(x_{i-1} , x_{i}) \leq \gamma_1 \); and
	\item \( z = x_{m} , x_{m+1} , \hdots , x_{n} = y \) with \( \max_{m+1 \leq i \leq n} d(x_{i-1} , x_{i}) \leq \gamma_2 \). 
\end{itemize}
Now set \( \gamma = \max \{ \gamma_1, \gamma_2 \}  \) and observe that \( x,y \) are \( \gamma \)-connected since \[ \max _{1 \leq i \leq n} d(x_{i-1} , x_{i} ) \leq \gamma = \frac{\gamma}{d(x,y)} \cdot d(x,y). \] 
Then we obtain
\begin{align*}
	c(x,y) \leq \gamma &= \max \{ \gamma_1, \gamma_2 \} \\
	       &\leq \max \{ c(x,z) + \varepsilon , c(z,y) + \varepsilon  \} \\
	       &= \max \{ c(x,z), c(z,y) \} + \varepsilon,
\end{align*}
so sending \( \varepsilon \to 0 \) yields \( c(x,y) \leq \max \{ c(x,z), c(z,y) \} \).
\end{proof}
\subsection{Properties of ultrametric spaces} The strong triangle inequality is much stronger than the usual triangle inequality. Because of this, ultrametric spaces have some elegant properties which we explore now.

First, it turns out that every triangle is isosceles.
\begin{lemma}
If \( x, y, z \) are distinct points in an ultrametric space \( X \) and \( d(x,y) < d(y,z) \) then \( d(y,z) = d(x,z) \).
\end{lemma}
\begin{proof}
Note that \( d \) is an ultrametric and \( d(x,y) < d(y,z) \) so that
\begin{align*}
	d(x, z) &\leq \max \{ d(x,y), d(y, z) \} = d(y,z),
\end{align*}
and since \( d(y,z) > d(x,y) \), we have \( d(x,y) < d(x,z) \) as otherwise
\begin{align*}
	d(y,z) &\leq \max \{ d(y, x) , d(x, z) \} = \max \{ d(x,y) , d(x,z) \} = d(x,y)
\end{align*}
is a contradiction. Thus \( d(y,z) \leq \max \{ d(x,y), d(x,z) \} = d(x,z) \). Combining everything together, we obtain \( d(y,z) = d(x,z) \).
\end{proof}

We now examine open balls in ultrametric spaces. An \emph{open ball} (or simply a ball) of radius \( \varepsilon > 0 \) centered about a point \( x \in  X \) is the set \[ B_{\varepsilon }(x) \coloneqq \{ y \in X : d(x,y) < \varepsilon  \}.  \] A subset \( \mathcal{O}  \) of \( X \) is called \emph{open} if it can be written as a union of open balls. A set is \emph{closed} if its complement is open.

Open balls in ultrametric spaces have vastly unintuitive properties. For example, every point in a ball is its center.

\begin{lemma}
Let \( B_{\varepsilon } (x) \) be an open ball in \( X \). Then \( B_{\varepsilon } (x) = B_{\varepsilon } (y) \) for every point \( y \in B_{\varepsilon } (x) \).
\end{lemma}
\begin{proof}
Fix \( y \in B_{\varepsilon } (x) \) so that \( d(x,y) < \varepsilon  \). If \( t \in B_{\varepsilon } (y) \) then \( d(y,t) < \varepsilon  \). Then \( t \in B_{\varepsilon } (x) \) since \[ d(x, t) \leq \max \{ d(x,y), d(y,t) \} < \varepsilon . \] The reverse inclusion follows symmetrically as \( x \in B_{\varepsilon } (y) \).
\end{proof}
Another important result is that if two balls intersect then one of them contains the other. More generally, we have the following lemma.
\begin{lemma}
Fix \( \varepsilon_1, \varepsilon_2 > 0 \) and \( x, y \in X \). If \( B_{\varepsilon_1} (x) \cap B_{\varepsilon_2}(y) \neq \emptyset  \) then \( B_{\varepsilon_1} (x) \subseteq B_{\varepsilon_2} (y) \) or \( B_{\varepsilon_{2} }(y) \subseteq B_{\varepsilon_1} (x)  \). (That is, two balls are either disjoint or one of them contains the other.)
\end{lemma}
\begin{proof}
Assume without loss of generality that \( \varepsilon_1 \leq \varepsilon_2 \). If \( B_{\varepsilon_1} (x) \cap B_{\varepsilon_2} (y) \neq \emptyset \), there is a point \( t \in X \) with \( t \in B_{\varepsilon_1}(x) \) and \( t \in  B_{\varepsilon_2} (y)  \). By (lemma) and since \( \varepsilon_1 \leq \varepsilon_2 \), we obtain \( B_{\varepsilon_1} (x) = B_{\varepsilon_1} (t) \subseteq B_{\varepsilon_2} (t) = B_{\varepsilon_2} (y) \) as required.
\end{proof}
A generalization of (lemma) is as follows.
\begin{lemma}
Let \( A \subseteq X \) and consider any ball \(B_{\varepsilon } (x) \) in \( X \). If \( B_{\varepsilon } (x) \) meets \( A \) then \( A \cap B_{\varepsilon } (x) \) is a ball in the space \( (A, d) \).
\end{lemma}
\begin{proof}
Since \( A \cap B_{\varepsilon } (x) \neq \emptyset  \), there is a point \( a \in A \cap B_{\varepsilon } (x) \). Let \( B_{A} = \{ t \in A : d(t,a) < \varepsilon  \}  \) be a ball in \( A \). We show that \( A \cap B_{\varepsilon } (x) = B_{A}  \).

Note from (lemma) that \( a \) is the center of \( B_{\varepsilon } (x) \) so that \( A \cap B_{\varepsilon } (x) = A \cap B_{\varepsilon } (a) \). Then if \( t \in A \cap B_{\varepsilon } (x) \) we have \( t \in A \cap B_{\varepsilon } (a) \) so that \( d(t, a) < \varepsilon  \) and hence \( t \in B_{A} \). On the other hand, if \( t \in B_{A}  \) then from the strong triangle inequality we obtain \[ d(x, t) \leq \max \{ d(x,a), d(a,t) \} < \varepsilon , \] since \( a \in B_{\varepsilon } (x) \). Hence, \( t \in A \cap B_{\varepsilon } (x) \) completes the proof.
\end{proof}
\begin{comment}
Since \( A \cap B \neq \emptyset \) there is a point \( a \in A \cap B \). Write \( B = B_{\varepsilon } (x) \) for some \( x \in X \) and \( \varepsilon > 0 \). To show that \( A \cap B \) is a ball in \( A \), we must find a radius \( r > 0 \) such that for every \( t \in A \cap B\), \( d(a,t) < r \). Set \( r = \varepsilon  \) and fix \( t \in A \cap B \). Then \( a, t \in B \) so that \( d(a,x) < \varepsilon  \) and \( d(t,x) < \varepsilon  \). Using the strong triangle inequality, \[ d(a,t) \leq \max \{ d(a,x) , d(t,x) \} < \varepsilon = r. \] Therefore, \( A \cap B \) is a ball in \( A \).
\end{comment}

Furthermore, it turns out that all open balls are also closed in \( X \).
\begin{lemma}
If \( B_{\varepsilon } (x) \) is an open ball in \( X \), then \( X \setminus  B_{\varepsilon } (x) \) is closed.
\end{lemma}
\begin{proof}
Suppose for a contradiction that there is a point \( t \in X \setminus B_{\varepsilon} (x) \) such that for every \( r > 0 \) the ball \( B_{r}(t) \) is not contained in \( X \setminus B_{\varepsilon } (x) \), so \( B_{r} (t) \cap B_{\varepsilon} (x) \neq \emptyset \). Setting \( r = \varepsilon  \) and using (lemma), we have \( B_{\varepsilon} (t) \subseteq B_{\varepsilon } (x) \) or \( B_{\varepsilon} (t) \subseteq B_{\varepsilon} (x) \). In either case, (lemma) implies equality. But then \( t \in X \setminus B_{\varepsilon } (x) = X \setminus B_{\varepsilon } (t) \) is a contradiction. Hence \( X \setminus B_{\varepsilon } (x) \) is a union of open balls so that \( B_{\varepsilon } (x) \) is closed in \( X \).
\end{proof}
Our final application of the ultrametric inequality is to the diameter of subsets of the space.
\begin{lemma}
Let \( A \subseteq X \) be non-empty with \( a \in A \). Then \( \operatorname{diam} A = \sup \{ d(a, x) : x \in A \}   \).
\end{lemma}
\begin{proof}
	Set \( u = \sup \{ d(a,x) : x \in A \}  \) and fix \( x,y \in A \). Then \[ d(x,y) \leq \max \{ d(x, a) , d(a, y) \} = \max \{ d(a,x), d(a,y) \} \leq u  \] so that \( u \) is an upper bound of \( D = \{ d(x,y) : x,y \in A \}  \). Since \( u = \sup_{} \{ d(a,x) : x \in A \}  \), for any given \( \varepsilon > 0 \), there exists a point \( x_{\varepsilon } \in A \) with \( u \leq d(a, x_{\varepsilon }) + \varepsilon   \). But \( a, x_{\varepsilon } \in A \) so that \( u = \sup D = \operatorname{diam} A \) which completes the proof.
\end{proof}
\begin{comment}
\subsection{Spherical Completeness} The space \( (X,d) \) is called \emph{spherically complete} if every decreasing sequence of closed balls has a non-empty intersection; that is, if \( \{ B_{n}  \}_{n = 1} ^{\infty}  \) is a sequence of closed balls in \( X \) with \( B_{k+1}  \subseteq B_{k}  \) for each \( k \in \mathbb{N}  \), then \( \medcap_{n=1}^{\infty}  B_{n} \neq \emptyset  \). For example, all compact metric spaces are spherically complete since they have the finite intersection property (cite).

It turns out that the strong triangle inequality coupled with spherical completeness yield some elegant results regarding best approximations [1]. In the following lemma, we extend the approach used in these findings to investigate the diameters of sets in spherically complete ultrametric spaces.

\begin{lemma}
Let \( (X,d) \) be a spherically complete ultrametric space with a non-empty bounded subset \( A \subseteq X \). Then \( A \) admits a pair of points \( a,b \in A \) with \( d(a,b) = \operatorname{diam} A \).
\end{lemma}
\begin{proof}
Since \( A \) is non-empty and bounded, there is a point \( a \in A \) and the quantity \( \operatorname{diam} A < \infty \). For each \( n \in \mathbb{N}  \), define \[ B_{n} = \{ x \in A : d(a,x) \leq \operatorname{diam} A + 1/n \}. \] Then \( \{ B_{n}  \}_{n = 1} ^{\infty}  \) is a decreasing sequence of balls, so spherical completeness yields a point \( b \in \medcap_{n=1} ^{\infty} B_{n}  \).

From (lemma), we have \( \operatorname{diam} A = \sup \{ d(a,x) : x \in A \}   \). It follows that \( d(a,b) = \operatorname{diam} A \). Indeed, for each \( x \in A \), \[ d(a,x) \leq \max \{ d(a,b), d(b,x) \}  \]

\end{proof}
\end{comment}


%! TeX root: ../main.tex
\section{Filters and Ultrafilters}
In this short section, we introduce a sequence of lemmas building up to Lemma \ref{filter:3}, which is fundamental to the proof of the main result. We start with some elementary examples of filters and ultrafilters, and we conclude with Lemma \ref{filter:4}, which is yet another elegant application of Lemma \ref{filter:3}. The development of these lemmas is due to the theory covered in the notes of Koppelberg \cite{notes:2011} and the work of Brian \cite{brian:2016}.

Recall from the introduction that a \emph{filter} \( \mathcal{F} \) on \( X \) is a family of subsets of \( X \) with \( \emptyset \notin \mathcal{F} \), \( X \in \mathcal{F} \), and which is closed under the superset inclusion and finite intersections. A filter \( \mathcal{F} \) is an \emph{ultrafilter} if no filter properly contains it, and we call \( \mathcal{F} \) \emph{free} if \( \medcap \mathcal{F} = \emptyset \). 

It turns out that one needs the axiom of choice to prove that free ultrafilters exist in \( \mathbb{N} \). Likewise, Zorn's lemma is required to prove that every filter is contained in an ultrafilter. For a more in-depth treatment, the interested reader is encouraged to see Section 3 in \cite{notes:2018}.

We briefly look at some simple examples of filters and ultrafilters as presented in \cite{notes:2011}. Then, we will move on to the proof of Lemma \ref{filter:1}.


\begin{itemize}[leftmargin=0.8cm]
\item For a non-empty subset \( A \subseteq \mathbb{N} \), the family \( \mathcal{F} = \{ B \subseteq \mathbb{N} : A \subseteq B \} \) is a filter. Note that \( \mathcal{F} \) is not free since each set in \( \mathcal{F} \) contains \( A \). If \( A \) is a singleton, then \( \mathcal{F} \) is an ultrafilter \cite{notes:2018}.
\item A subset \( A \subseteq \mathbb{N} \) is called \emph{cofinite} if \( \mathbb{N} \setminus A \) is finite. The collection of cofinite subsets of \( \mathbb{N} \) is a filter, called the Fr\'echet filter. The Fr\'echet filter is not an ultrafilter, but it is free since \( \mathbb{N} \setminus \{ m \} \) is cofinite for every integer \( m \in \mathbb{N} \). In fact, a filter is free if and only if it contains the Fr\'echet filter (see Proposition 2 in \cite{notes:2015}).
\item For a topological space \( X \) and a point \( x \in X \), we call a set \( N \subseteq X \) a \emph{neighbourhood} of \( x \) if there is an open subset of \( N \) containing \( x \). Then the family \( \mathcal{U}(x) \) of all neighbourhoods of \( x \) is a filter, and \( \mathcal{U}(x) \) is an ultrafilter if and only if \( \{ x \} \) is open in \( X \).
\end{itemize}

With these examples covered, we are ready for our first lemma.

\begin{lemma}
\label{filter:1}
If \( \mathcal{A} \) is a non-empty family of non-empty subsets of \( X \) such that any finite intersection of sets in \( \mathcal{A} \) is non-empty, then there is a filter \( \mathcal{F} \) which contains \( \mathcal{A} \).
\end{lemma}
\begin{proof}
First let \( \mathcal{F}' = \mathcal{A} \cup \mathcal{I} \), where \( \mathcal{I} \) is the set of all finite intersections of sets in \( \mathcal{A} \). Then \( \mathcal{F'} \) is closed under finite intersections. Then let \( \mathcal{F} = \mathcal{F}' \cup \mathcal{S} \), where \( A \in \mathcal{S} \) if and only if \( A \) contains a set in \( \mathcal{F}' \).

We first note that \( \emptyset \notin \mathcal{F} \) since the intersection of any finite sub-collection of \(\mathcal{A} \) is non-empty. Likewise, \( X \in \mathcal{F} \) by the non-emptyness of \( \mathcal{A} \) and the construction of \( \mathcal{S} \). Clearly \( \mathcal{F} \) is closed under the superset inclusion. It remains to show that if \( A, B \in \mathcal{F} \) then \( A \cap B \in \mathcal{F} \). 

If \( A, B \in \mathcal{F} \), the only non-trivial case needing consideration is, without loss of generality, when \( A \in \mathcal{S} \). So there are sets \( A', B' \in \mathcal{F}' \) with \( A' \subseteq A \) and \( B' \subseteq B \) (if \( B \in \mathcal{F}' \) then \( B' = B \)). Since \( \mathcal{F} ' \) is closed under finite intersections, \( A' \cap B' \in \mathcal{F} ' \). Then, from \( A' \cap B' \subseteq A \cap B \) it follows that \( A \cap B \in \mathcal{S} \subseteq \mathcal{F} \). Hence \( \mathcal{F} \) is closed under finite intersections, as needed.
\end{proof}
\begin{lemma}
\label{filter:2}
A filter \( \mathcal{F} \) is an ultrafilter if and only if for every subset \( A \subseteq X \) either \( A \in \mathcal{F} \) or \( A^{c} \in \mathcal{F} \).
\end{lemma}
\begin{proof}
For ``$\Rightarrow$'', let \( \mathcal{F} \) be an ultrafilter and suppose for a contradiction that there is a subset \( A \subseteq X \) with \( A \notin \mathcal{F} \) and \( A^{c} \notin \mathcal{F} \). So every set in \( \mathcal{F} \) intersects both \( A \) and \( A^{c} \). Using this and the fact that \( \mathcal{F} \) is a filter, it follows that any finite sub-collection of \( \mathcal{F} \cup \{ A \} \) has a non-empty intersection. Then, Lemma \ref{filter:1} implies that there is a filter containing \( \mathcal{F} \cup \{ A \} \), and so \( \mathcal{F} \) is properly contained in a filter, a contradiction.

For ``\(\Leftarrow\)", suppose \( \mathcal{F} \) is a filter with the property that \( A \in \mathcal{F} \) or \( A^{c} \in \mathcal{F} \) for every subset \( A \subseteq X \). If \( \mathcal{F} \) is not an ultrafilter then there is another filter \( \mathcal{F} ' \) which properly contains \( \mathcal{F} \). Hence there is a subset \( E \subseteq X \) with \( E \in \mathcal{F} ' \) and \( E \notin \mathcal{F} \). Thus \( E^{c} \in \mathcal{F} \) and hence \( E^{c} \in \mathcal{F} ' \). But then \( \mathcal{F} ' \) contains the empty set since it is closed under finite intersections and \( \emptyset = E \cap E^{c} \in \mathcal{F} ' \), which is a contradiction.
\end{proof}
We will say that a filter \( \mathcal{F} \) has the \emph{Ramsey property} if whenever \( \medcup_{j=1}^{n} A_{j} \in \mathcal{F} \) there is a \( j \in [n] \) with \( A_{j} \in \mathcal{F} \). In this connection, we have the following surprising result.
\begin{lemma}
\label{filter:3}
A filter \( \mathcal{F} \) is an ultrafilter if and only if \( \mathcal{F} \) has the Ramsey property.
\end{lemma}
\begin{proof}
It suffices to prove the claim for \( n = 2 \). Let \( \mathcal{F} \) be an ultrafilter and suppose for a contradiction that \( A = A_1 \cup A_2 \) is in \( \mathcal{F}\) but \( A_1 \notin \mathcal{F} \) and \( A_2 \notin \mathcal{F} \). From Lemma \ref{filter:2}, we have \( A_1^{c} \in \mathcal{F} \) and \( A_2^{c} \in \mathcal{F} \) so that \( A^{c} = A_1^{c} \cap A_2^{c} \in \mathcal{F} \). Consequently, \( \emptyset = A \cap A^{c} \in \mathcal{F} \) is a contradiction. 

Conversely, assume \( \mathcal{F} \) has the Ramsey property. If \( \mathcal{F} \) is not an ultrafilter, then Lemma \ref{filter:2} asserts that there is a subset \( A \subseteq X \) with \( A \notin \mathcal{F} \) and \( A^{c} \notin \mathcal{F} \). But \(X \in \mathcal{F} \) and \( X = A \cup A^{c} \), so we must have \( A \in \mathcal{F} \) or \( A^{c} \in \mathcal{F} \), contradicting our choice of \( A \).
\end{proof}

For a family \( \mathcal{F} \) of subsets of \( X \), we define its \emph{dual} \( \mathcal{F} ^{\ast} \) to be the collection of all subsets of \( X \) which intersect every set in \( \mathcal{F} \).

A simple example of families and their duals is as follows \cite{brian:2016}. We call a subset \( A \subseteq \mathbb{N} \) \emph{thick} if it contains intervals of arbitrary lengths, and we call \( A \) \emph{syndetic} if the space between its intervals is bounded. That is, \( A \) is syndetic if there is an integer \( N \in \mathbb{N} \) such that every interval of length \( N \) contains a point in \( A \). Then, the dual of the family of syndetic subsets of \( \mathbb{N} \) is the class of thick sets in \( \mathbb{N} \).

The following result, due to Glasner, relates ultrafilters to their duals \cite{glasner:1980}. We provide its proof here for clarity.

\begin{lemma}
\label{filter:4}
A filter \( \mathcal{F} \) is an ultrafilter if and only if \( \mathcal{F} ^{\ast} \) is a filter.
\end{lemma}
\begin{proof}
For ``$\Rightarrow$", assume that \( \mathcal{F} \) is an ultrafilter, and fix \( A,B \in \mathcal{F} ^{\ast} \). To prove that \( A \cap B \in \mathcal{F} ^{\ast} \), it suffices to fix a set \( E \in \mathcal{F} \) and prove that \( E \) intersects \( A \cap B \). Note that since \( A \in \mathcal{F} ^{\ast} \) we have \( E \cap A \neq \emptyset \). Hence we may write \[ E = (E \cap A) \cup (E \setminus A). \] Certainly \( E \setminus A \notin \mathcal{F} \), as otherwise \( A \in \mathcal{F} ^{\ast} \) implies \( (E\setminus A) \cap A \neq \emptyset \). Since \( \mathcal{F} \) is an ultrafilter, Lemma \ref{filter:3} implies that \( \mathcal{F} \) has the Ramsey property so that \( E \cap A \in \mathcal{F} \). Since \( B \in \mathcal{F} ^{\ast} \), it follows as needed that \( E \cap (A \cap B) \neq \emptyset \).

Conversely, for ``$\Leftarrow$", let \( \mathcal{F} ^{\ast} \) be a filter. Suppose for a contradiction that \( \mathcal{F} \) is not an ultrafilter. Then Lemma \ref{filter:2} implies that there is a subset \( A \subseteq X \) with \( A \notin \mathcal{F} \) and \( A^{c} \notin \mathcal{F} \). Thus, every set in \( \mathcal{F} \) intersects both \( A \) and \( A^{c} \). That is, for every \( B \in \mathcal{F} \) we have \( A \cap B \neq \emptyset \) and \( A^{c} \cap B \neq \emptyset \). By definition, then, \( A \in \mathcal{F}^{\ast} \) and \( A^{c} \in \mathcal{F} ^{\ast} \). But \( \mathcal{F} ^{\ast} \) is a filter, so \( \emptyset = A \cap A^{c} \in \mathcal{F} ^{\ast} \) is a contradiction, and the proof is complete.
\end{proof}

\section{Main Result}
\subsection{Some Lemmas and Preliminaries}
%% def ctbl === injects into N
\begin{lemma}
Let \( (X, d) \) be an infinite metric space. Then there is a sequence \( \{ x_{n}  \}_{n = 1} ^{\infty}  \) of distinct points in \( X \) such that either
\begin{enumerate}[leftmargin = 1.2cm]
	\item The sequence \( \{ d(x_1, x_{n} ) \}_{n = 1} ^{\infty}  \) is strictly monotone; or
	\item For every \( n \in \mathbb{N}  \) and \( i,j \geq n \) the distances \( d(x_{n} , x_{i} ) = d(x_{n} , x_{j} ) \).
	
\end{enumerate}
\end{lemma}
\begin{proof}
We first assume that there is a point \( x_0 \in X \) such that \( d(x_0, X) \coloneqq \{ d(x_1, x_{n} ) : x_{n} \in X \}  \) is not finite. Hence, there is a countably infinite subset \( E \subseteq X \) with \( x_0 \notin E \) and \( d(x_0, x) \neq d(x_0, y) \) for every \( x,y \in E \). We obtain from \( E \) the sequence \( \xi = \{ d(x_0, x) : x \in E \}  \). Since \( \xi \) is a sequence of reals, it has a monotone subsequence \( \{ d(x_0, x_{n} ) \}_{n=1} ^{\infty} \) whose points are distinct by construction of \( E \). Since \( d \) is a metric, \( x_{i} \neq x_{j}  \) for every \( i,j \in \mathbb{N}  \) and so \( \{ x_{n}  \}_{n = 1} ^{\infty}  \) is the desired sequence.

Otherwise, \( d(x, X) \) is finite for every \( x \in X \). Fix \( x_0 \in X \) and assume without loss of generality that \( \ell_1 \in d(x_0, X)\) is non-zero. Let \( E_1 \) be a countable subset of \( X \) with \(E_1 \subseteq  \{ x \in X : d(x_0, x) = \ell_1 \}. \) Choose \( x_1 \in E_1 \), and note that \( d(x_0, E_1) = \{ \ell_1 \} \) is a singleton set and \( x_0 \notin E_1 \). 

For \( n \geq 2 \), we choose \( x_{n}  \) and define \( E_{n}  \) inductively as follows. As above, let \( \ell_{n} \in d(x_{n-1} , X) \) be non-zero and let \( E_{n}  \) be a countable subset of \( X \) with \( E_{n} \subseteq \{ x \in X : d(x_{n-1} , x) = \ell_{n} \} . \) Again, we choose \( x_{n} \in E_{n}  \) and observe that \( d(x_{n-1} , E_{n} ) = \{ \ell_{n}  \}  \). Continuing this way, we obtain the sequence from (2).
\end{proof}

Set \( k \in \mathbb{N} \) and let \( X \) be a metric space. A \( k \)-\textbf{coloring} on \( X \) is a function \( \chi  : [X]^{k} \to [k] \), where \( [X]^{k} = \{ \{ x_1, x_2, \ldots, x_{k}  \} : x_{i} \in X, \ \forall i \in [k]\}   \) is the set of all \( k \) element subsets of \( X \). A subset \( A \subseteq X \) is called \( \chi \)-\textbf{monochrome} if \( \chi \) is constant on \( [A]^{k}  \). The coloring \( \chi \) is called \( k \)-\textbf{isometric} if \( \chi (A_1) = \chi (A_2)  \) whenever the pair \( A_1, A_2 \in [X]^{k}  \) of \( k \) element subsets satisfy \( \operatorname{diam}A_1 = \operatorname{diam}A_2  \).

A free ultrafilter \( \mathcal{F}  \) on \( X \) is called \( k \)-\textbf{Ramsey} with respect to a collection \( \mathcal{C}  \) of colorings on \( X \) if for every coloring \( \chi \in \mathcal{C}  \) there is a set \( A  \in \mathcal{F}  \) such that \( [A]^{k}  \) is \( \chi \)-monochrome.
\subsection{Main Result}
(Authors) in [3] prove that every free ultrafilter on an infinite ultrametric space \( X \) is \( 2 \)-Ramsey with respect to the class of \( 2 \)-isometric colorings on \( X \). In this particular case, a coloring \( \chi \) is \( 2 \)-isometric if and only if all points \( x_1, x_2, y_1, y_2 \in X \) with \( d(x_1, y_1) = d(x_2, y_2) \) satisfy \( \chi (\{ x_1, y_1 \}) = \chi (\{ x_2, y_2 \} ) \). (Authors) leverage the properties of the ultrametric coupled with this observation and (lemma) to prove the main result when \( k = 2 \). We expand on this approach, strengthening their result to hold for all \( k \)-colorings in the particular case of ultrametric spaces.

\begin{theorem}
Let \( (X,d) \) be an infinite ultrametric space and let \( k \) be a positive integer. Let \(E = \{ c_{n}  \}_{n = 1} ^{\infty}  \) be a sequence of points obtained as in (lemma). Then every free ultrafilter \( \mathcal{F}  \) in \( X \) containing \( E \) is \( k \)-Ramsey with respect to the collection \( \mathcal{C}  \) of \( k \)-isometric colorings on \( X \).
\end{theorem}
\begin{proof}
Let \( \chi \in \mathcal{C}  \) be a \( k \)-isometric coloring on \( X \) and fix a free ultrafilter \( \mathcal{F}  \) which contains \( E \).

Let \( h  : E \to \mathbb{R^{+}} \) be a fixed map. How we define \( h \) will depend on the sequence \( E \) attained from (lemma), so we will proceed in this regard later on. Moreover, suppose \( f : h (E) \to [k] \) is any mapping satisfying \( f(h(c_{\ell} )) = \chi (A) \) whenever \( A \in [X]^{k}  \) is a \( k \) element subset of \( X \) with \( \operatorname{diam}A = h(c_{\ell} ) \). Finally, we set \( c = f \circ h  \).

Write \( E = c ^{-1} ([k]) = \cup_{j=1}^{k} c^{-1} (\{ j \} )  \) and observe that since \( E \in \mathcal{F}  \), (lemma) implies that there is a color \( \varphi \in [k] \) with \( c ^{-1} (\{ \varphi  \} ) \in \mathcal{F}   \). We set \( A = c ^{-1} (\{ \varphi  \} )  \) and complete the proof by showing that \( A \) is \( \chi \)-monochrome. In particular, we will show that each \( k \) element set in \( [A]^{k}  \) has color \( \varphi \).




Let \( n_1 < n_2 < \cdots < n_{k}  \) be fixed positive integers and consider the \( k \) element subsequence \(C_{k} = \{ c_{n_1} , c_{n_2} , \hdots , c_{n_{k} }   \} \in [A]^{k} \) of \( E \). Assume integers \( n_{i} < n_{j}  \) are such that \[ \{ c_{n_{i} } , c_{n_{j} }  \} = \underset{\{ x, y \} \subseteq C_{k}  }{\operatorname{arg\,max}} \ d(x,y); \] that is, \( d(c_{n_{i} } , c_{n_{j} }) = \operatorname{diam}C_{k}  \). We now consider the conditions on \( E \) as described in (lemma), and define \( h \) accordingly to complete the proof.

\textbf{Case 1.} We will first assume that case (1) of (lemma) holds, namely that \( \{ c_{n}  \} _{n=1} ^{\infty}  \) is a sequence of distinct points in \( X \) where \( \{ d(c_0, c_{n} )  \}_{n = 1} ^{\infty} \) is strictly monotone. In this case, \( h \) will indicate the distance to \( c_0 \) from a term \( c_{\ell} \in E \), given by \( h (c_{\ell} ) = d(c_0, c_{\ell} ) \).

If \( h  \) is strictly increasing, then we have \( d(c_0, c_{n_{i} } ) < d(c_0, c_{n_{j} } ) \). Hence (lemma) implies that \( d(c_{n_{i} } , c_{n_{j} } ) = d(c_0, c_{n_{j} } ) \), since \( d \) is an ultrametric. Otherwise \( d(c_0, c_{n_{i} } ) > d(c_0, c_{n_{j} }) \) so that \( d(c_{n_{i} } , c_{n_{j} } ) = d(c_0, c_{n_{i} } ) \) using (lemma) once more. Possibly swapping the symbols \( i,j \), we assume the latter case holds. Since \( c_{n_{j} } \in A \) and \( \operatorname{diam}C_{k} = d(c_{n_{i} } , c_{n_{j} } ) = d(c_0, c_{n_{j} } )  \), we have \( f(d(c_0, c_{n_{j} } )) = \chi (C_{k}) = \varphi \), as needed.

\textbf{Case 2.} We now assume that case (2) of (lemma) applies to \( \{ c_{n}  \}_{n = 1} ^{\infty}  \). Namely, for each \( n \in \mathbb{N}  \) and \( i,j \geq n \) we have \( d(c_{n} , c_{i} ) = d(c_{n} , c_{j} ) \). Define \( h : E \to \mathbb{R}^{+}  \) by \( h(c_{\ell}) = d(c_{\ell} , c_{\ell + 1}) \), and note that \( h(c_{\ell}) = d(c_{\ell} , c_{j})  \) for every \( j > \ell \). Since \( n_{i} < n_{j}  \), we have \( h(c_{n_{i} }) = d(c_{n_{i} } , c_{n_{j} } )\) and, as desired, \[ f(h(c_{n_{i} } )) = f(d(c_{n_{i} } , c_{n_{j} } )) = \chi (C_{k}) = \varphi. \]

This completes the proof, since \( A \) is \( \chi \)-monochrome in both cases.
\end{proof}
\end{document}
