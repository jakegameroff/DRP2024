%! TeX root: ../main.tex
\section{Introduction}
Ramsey Theory explores the underlying structure emerging in ``large enough" complex systems. For example, Frank Ramsey [1] proved that for each \( n \in \mathbb{N}  \) there is a sufficiently large \( N \in \mathbb{N}  \) such that in any red-blue coloring of the edges of the complete graph \( K_{N} \) there is a set of \( n \) vertices joined by pairwise monochromatic edges (i.e., a monochromatic clique).  This raises a natural question: given \( n, k \in \mathbb{N}  \), is there an integer \( N \in \mathbb{N} \) such that in any edge-coloring of \( K_{N}  \) in \( k \) colors there is a monochromatic clique of size \( n \)? Its positive answer is due to (authors) [2]. Analogously, in what follows, we generalize and strengthen a Ramsey-type coloring theorem of Protasov and Protasova [3] to hold for $k$ colors, where \( k \in \mathbb{N}  \).

Fix an infinite metric space \( (X,d) \) and let \( k \in \mathbb{N}  \). A \emph{\( k \)-coloring} on \( X \) is a map \( \chi : [X]^{k} \to [k]  \) which assigns one of \( k \) colors to each \( k \)-element subset of \( X \). For a given \( k \)-coloring \( \chi \), we would like to find a set \( A \subseteq X \) such that \( \chi ([A]^{k}) = \{ c \}  \) for some \( c \in [k] \); in this case, we call \( [A]^{k}  \) \emph{\( \chi \)-monochrome}.

In this context, the ``large" objects containing underlying structure in \( (X,d) \) are free ultrafilters [4]. A \emph{filter} \( \mathcal{F}  \) on \( X \) is a collection of subsets of \( X \) satisfying the following for all sets \( A, B \subseteq X \):
\begin{enumerate}[leftmargin=1.2cm]
	\item \( \emptyset  \notin \mathcal{F}  \) and \( X \in \mathcal{F}  \);
	\item If \( A \in \mathcal{F} \) and \( A \subseteq B \) then \( B \in \mathcal{F}  \);
	\item If \( A, B \in \mathcal{F}  \) then \( A \cap B \in \mathcal{F}  \).
\end{enumerate}
A filter \( \mathcal{F}  \) is called an \emph{ultrafilter} if it is not properly contained in a filter on \( X \). A filter \( \mathcal{F}  \) is called \emph{free} if \( \medcap \mathcal{F} = \emptyset  \). Free filters are ``spread out" throughout the space and ultrafilters are maximal filters, so we consider \emph{free ultrafilters} as ``large" objects in \( X \).

Given this, one naturally asks if there is a free ultrafilter \( \mathcal{F}  \) such that for every coloring \( \chi \) there is a set \( A \in \mathcal{F}  \) so that \( [A]^{k}  \) is \( \chi \)-monochrome. It turns out that this question is undecidable in ZFC even with \( X = \mathbb{N}  \) and \( d(x,y) = |x-y| \), though the statement is true if we declare the continuum hypothesis as axiomatic [5]. Hence we must define more restrictive classes of colorings.

To this end, we say that a \( k \)-coloring \( \chi : [X]^{k} \to [k]  \) is an \emph{isometric \( k \)-coloring} if \( \chi (A_1) = \chi (A_2) \) whenever \( A_1, A_2 \) is a pair of isodiametric \( k \)-element subsets of \( X \). A free ultrafilter \( \mathcal{F}  \) is called \emph{metrically Ramsey} if for every isometric \( k \)-coloring \( \chi \) there is a set \( A \in \mathcal{F}  \) such that \( [A]^{k}  \) is \( \chi \)-monochrome.

Recall that an \emph{ultrametric} \( d \) on a set \( X \) is a metric satisfying the \emph{strong triangle inequality:} for all \( x,y,z \in X \) \[ d(x,y) \leq \max \{ d(x,z), d(z,y) \}.  \]
It turns out that in the particular case of ultrametric spaces, metrically Ramsey free ultrafilters are not too hard to construct. Indeed, Protasov and Protasova [6] prove that for \( k = 2 \) every infinite ultrametric space \( X \) contains a sequence \( (x_{n}) \) such that every free ultrafilter containing \( (x_{n}) \) is metrically Ramsey. The authors leverage the properties of the ultrametric coupled with (lemma) (to construct \( (x_{n}) \)) to prove the main result when \( k = 2 \). We expand on this approach, strengthening their result to hold for all \( k \)-colorings.

In the following two sections, we review some elementary properties of ultrametrics and filters. We then prove the main result in (section 4).

