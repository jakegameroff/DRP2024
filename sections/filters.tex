%! TeX root: ../main.tex
\section{Filters and Ultrafilters}
In this short section, we introduce a sequence of lemmas building up to Lemma \ref{filter:3}, which is fundamental to the proof of the main result. We then conclude the section with Lemma \ref{filter:4}, which is yet another elegant application of Lemma \ref{filter:3}. The development of these lemmas was informed by concepts discussed in the notes of Koppelberg \cite{notes:2011} and the work of Brian \cite{brian:2016}.

Recall from the introduction that a \emph{filter} \( \mathcal{F}  \) on \( X \) is a family of subsets of \( X \) with \( \emptyset \notin \mathcal{F}  \), \( X \in \mathcal{F}  \), and which is closed under the superset inclusion and finite intersections. A filter \( \mathcal{F}  \) is an \emph{ultrafilter} if no filter properly contains it, and we call \( \mathcal{F}  \) \emph{free} if \( \medcap \mathcal{F} = \emptyset  \).

The first lemma is the following.
\begin{lemma}
\label{filter:1}
If \( \mathcal{A}  \) is a non-empty family of subsets of \( X \) such that any intersection of finitely many sets in \( \mathcal{A}   \) is non-empty, then there is a filter \( \mathcal{F}  \) which contains \( \mathcal{A} \).
\end{lemma}
\begin{proof}
First let \( \mathcal{F}' =  \mathcal{A} \cup \mathcal{I} \), where \( \mathcal{I} \) is the set of all finite intersections of elements of \( \mathcal{A}   \). Then \( \mathcal{F'}  \) is closed under finite intersections. Then let \( \mathcal{F} = \mathcal{F}' \cup \mathcal{S}  \), where \( A \in \mathcal{S}  \) if and only if \( A \) contains a set in \( \mathcal{A} \cup \mathcal{I}  \).

We first note that \( \emptyset \notin \mathcal{F} \) since the intersection of any finite sub-collection of \(\mathcal{A}   \) is non-empty. Likewise, \( X \in \mathcal{F}  \) by the non-emptyness of \( \mathcal{A}  \) and the construction of \( \mathcal{S}  \). Clearly \( \mathcal{F}  \) is closed under the superset inclusion. It remains to show that if \( A, B \in \mathcal{F}  \) then \( A \cap B \in \mathcal{F}  \). 

If \( A, B \in \mathcal{F}  \), the only non-trivial case needing consideration is, without loss of generality, when \( A \in \mathcal{S}  \). Then there exist sets \( A', B' \in \mathcal{F}' \) with \( A' \subseteq A \) and \( B' \subseteq B \) (if \( B \in \mathcal{F}' \) then \( B' = B \)). Since \( \mathcal{F} ' \) is closed under finite intersections, \( A' \cap B' \in \mathcal{F} ' \). Then, from \( A' \cap B' \subseteq A \cap B \) it follows that \( A \cap B \in \mathcal{S} \subseteq \mathcal{F}  \). Hence \( \mathcal{F}  \) is closed under finite intersections, as needed.
\end{proof}
\begin{lemma}
\label{filter:2}
A filter \( \mathcal{F}  \) is an ultrafilter if and only if for every subset \( A \subseteq X \) either \( A \in \mathcal{F}  \) or \( A^{c} \in \mathcal{F}  \).
\end{lemma}
\begin{proof}
For ``$\Rightarrow$'', let \( \mathcal{F}  \) be an ultrafilter and suppose for a contradiction that there is a subset \( A \subseteq X \) with \( A \notin \mathcal{F}  \) and \( A^{c} \notin \mathcal{F}  \). So every set in \( \mathcal{F}  \) intersects both \( A \) and \( A^{c}  \). Using this and the fact that \( \mathcal{F}  \) is a filter, it follows that any finite sub-collection of \( \mathcal{F} \cup \{ A \}  \) has a non-empty intersection. Then, Lemma \ref{filter:1} implies that there is a filter containing \( \mathcal{F} \cup \{ A \}  \) and hence \( \mathcal{F}  \), which contradicts our choice of \( \mathcal{F}  \).

Now suppose for ``$\Leftarrow$" that \( \mathcal{F}  \) is a filter such that for every subset \( A \subseteq X \) either \( A \in \mathcal{F}  \) or \( A^{c} \in \mathcal{F}  \). Assume for a contradiction that \( \mathcal{F}' \) is a filter which properly contains \( \mathcal{F}  \). So there is a subset \( E \subseteq X \) with \( E \in \mathcal{F} ' \) and \( E \notin \mathcal{F}  \). Thus \( E^{c} \in \mathcal{F}  \) and hence \( E^{c} \in \mathcal{F} ' \). But then \( \mathcal{F} ' \) contains the empty set since it is closed under finite intersections and \( \emptyset = E \cap E^{c} \in \mathcal{F} ' \), which is a contradiction.
\end{proof}
We will say that a filter \( \mathcal{F}  \) has the \emph{Ramsey property} if whenever \( \medcup_{j=1}^{n} A_{j}  \in \mathcal{F}  \) there is a \( j \in [n] \) with \( A_{j} \in \mathcal{F}  \). In this connection, we have the following surprising result.
\begin{lemma}
\label{filter:3}
A filter \( \mathcal{F}  \) is an ultrafilter if and only if \( \mathcal{F}  \) has the Ramsey property.
\end{lemma}
\begin{proof}
	Let \( \mathcal{F}  \) be an ultrafilter and suppose for a contradiction that \( A = A_1 \cup A_2 \) is in \( \mathcal{F}\) but \( A_1, A_2 \notin \mathcal{F}  \). From Lemma \ref{filter:2}, we have \( A_1^{c} , A_2^{c} \in \mathcal{F}  \) so that \( A^{c} = A_1^{c} \cap A_2^{c} \in \mathcal{F}  \). Consequently, \( \emptyset = A \cap A^{c} \in \mathcal{F}  \) is a contradiction. The case where \( A = A_1 \cup \cdots \cup A_{n}  \) follows via elementary induction. Conversely, assume \( \mathcal{F}  \) has the Ramsey property. If \( \mathcal{F}  \) is not an ultrafilter, then there is a subset \( A \subseteq X \) with \( A \notin \mathcal{F}  \) and \( A^{c} \notin \mathcal{F}  \). But \(X \in \mathcal{F}  \) and \( X = A \cup A^{c}  \), so we must have \( A \in \mathcal{F}  \) or \( A^{c} \in \mathcal{F}  \), contradicting our choice of \( A \).
\end{proof}

For a family \( \mathcal{F}  \) of subsets of \( X \), we define its \emph{dual} \( \mathcal{F} ^{\ast}  \) to be the collection of all subsets of \( X \) which intersect every set in \( \mathcal{F}  \).

A simple example of families and their duals is as follows \cite{brian:2016}. We call a subset \( A \subseteq \mathbb{N}  \) \emph{thick} if it contains intervals of arbitrary lengths; and we call \( A \) \emph{syndetic} if the space between its intervals is bounded. That is, \( A \) is syndetic if there is an integer \( N \in \mathbb{N}  \) such that every interval of length \( N \) contains a point in \( A \). Then, the dual of the family of syndetic subsets of \( \mathbb{N} \) is the class of thick sets in \( \mathbb{N}  \).

The following result, due to Glasner, relates ultrafilters to their duals \cite{glasner:1980}. We provide its proof here for clarity.

\begin{lemma}
\label{filter:4}
A filter \( \mathcal{F}  \) is an ultrafilter if and only if \( \mathcal{F} ^{\ast}  \) is a filter.
\end{lemma}
\begin{proof}
	For ``$\Rightarrow$", assume that \( \mathcal{F}  \) is an ultrafilter, and fix \( A,B \in \mathcal{F} ^{\ast}  \). To prove that \( A \cap B \in \mathcal{F} ^{\ast}  \), it suffices to fix a set \( E \in \mathcal{F}  \) and prove that \( E \) intersects \( A \cap B \). Note that since \( A \in \mathcal{F} ^{\ast}  \) we have \( E \cap A \neq \emptyset  \). Hence we may write \[ E = (E \cap A) \cup (E \setminus A). \] Certainly \( E \setminus A \notin \mathcal{F}  \), as otherwise \( A \in \mathcal{F} ^{\ast}  \) implies \( (E\setminus A) \cap A \neq \emptyset \). Since \( \mathcal{F}  \) is an ultrafilter, Lemma \ref{filter:3} implies that \( \mathcal{F}  \) has the Ramsey property so that \( E \cap A \in \mathcal{F}  \). Since \( B \in \mathcal{F} ^{\ast}  \), it follows as needed that \( E \cap A \cap B \neq \emptyset  \).

	Conversely, for ``$\Leftarrow$", let \( \mathcal{F}  \) and \( \mathcal{F} ^{\ast}  \) be filters. Suppose towards a contradiction that \( \mathcal{F}  \) is not an ultrafilter. Then Lemma \ref{filter:2} implies that there is a subset \( A \subseteq X \) with \( A \notin \mathcal{F}  \) and \( A^{c} \notin \mathcal{F}  \). Thus, every set in \( \mathcal{F}  \) intersects both \( A \) and \( A^{c}  \). That is, for every \( B \in \mathcal{F}  \) we have \( A \cap B \neq \emptyset  \) and \( A^{c} \cap B \neq \emptyset  \). By definition, then, \( A \in \mathcal{F}^{\ast}   \) and \( A^{c} \in \mathcal{F} ^{\ast}  \). But \( \mathcal{F} ^{\ast}  \) is a filter, so \( \emptyset = A \cap A^{c} \in \mathcal{F} ^{\ast}  \) is a contradiction, and the proof is complete.
\end{proof}
